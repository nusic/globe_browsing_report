\chapter{Introduction}

Scientific visualization of space research, also known as astro-visualization, becomes increasingly important for scientists to communicate their work in exploring the cosmos. 3D computer graphics has shown to be an efficient tool for bringing insights from geological and astronomical data, as spatial and temporal relations can intuitively be interpreted through 3D visualizations.

Researching and mapping out celestial bodies other than the Earth is an important part of expanding the space frontier and virtually rendering these globes using real gathered map and terrain data is a natural part of any scientifically accurate space visualization software.

Two important parts of a software for visualizing celestial bodies are terrain and atmosphere. The focus of this thesis is put on terrain rendering on globes using high fidelity geographical data such as texture maps and digital terrain models. The globe rendering feature with the research involved was implemented for the software OpenSpace. The implementation was separated enough from the main program to avoid dependencies and make the thesis independent of specific implementation details. 

\section{Background}

\subsection{OpenSpace}

OpenSpace is an open-source, interactive data visualization software with the goal of bringing astro-visualization to the broad public and serve as a platform for scientists to talk about their research. The software supports rendering across multiple screens, allowing immerse visualizations on tiled displays as well as in dome theatres \cite{openspace}.

With a real time rendering software such as OpenSpace, the human curiosity involved in exploration easily becomes obvious when the user is given the ability to freely fly around in space and near the surface of other worlds and discover places they probably never can visit in real life. This is also true for public presentations where researchers such as geologists can go into details about their knowledge in these places.

An important part of the software is to avoid the use of procedurally generated data. This is to express where the frontier of science and exploration is currently at, and how it progresses through space missions with the goal of mapping the Universe. A general globe browsing feature provides a means of communicating this progress through continuous mapping of our planets and moons.

\subsection{Globe Browsing}

The term globe browsing can be described as exploration of geospatial data on a virtual representation of a globe. The word globe is a general term used to describe nearly elliptical celestial objects such as planets, moons and asteroids.

Globe rendering with the purpose of multi scale browsing has been used for quite some time in flight simulators, map services and astro-visualization. Prerendered flight paths were visualized as early as the late 1970s by NASA's Jet Propulsion Laboratory \cite{cozzi11}. 

Google Earth \cite{googlemaps} enables browsing of the Earth within a web browser using geometries for cities of high detail. The National Oceanic and Atmospheric Administration (NOAA) provides a sophisticated sphere rendering system, ''Science On a Sphere'', with the ability to visualize a vast amount of geospatial data on spheres with a temporal dimension \cite{sos}. The project is not only a software but includes projection displays and floor plans to aid in public presentations.

There are other commercial softwares that enables larger scale visualization of the Universe with real positional data gathered through research by the National Aeronautics and Space Administration (NASA), European Space Agency (ESA) and others. Satellite Toolkit (STK) enables this by integrating ephemeris information through the SPICE interface \cite{spice} which allows accurate placing of celestial bodies and space crafts within our solar system. Uniview from SCISS AB also enables SPICE integration with sophisticated rendering techniques and dome theatre support \cite{uniview}.

There are other significant globe browsing softwares used in dome theatres such as Microsoft's World Wide Telescope (WWT) \cite{wwt}, Evans \& Sutherland's Digistar \cite{digistar}, and Sky-Skans DigitalSky \cite{digitalsky}.

Other relevant softwares that currently do not support dome configuration rendering but none the less are very adequate in their techniques of integrating globe browsing and globe rendering include Outerra \cite{outerra} and Space Engine \cite{spaceengine}. Both focusing on merging real data with procedurally generated terrains where real data is not available.

Geographic information systems (GIS) are softwares with the purpose of gathering a wide range of geographic map data and visualizing it in various different ways. Even though most of these softwares use GIS features, many of them are not considered GIS. However, they all have the globe browsing feature in common. Technicalities in how it is implemented varies as their end target users are different.

In table \ref{table:softwares}, features relevant to globe browsing in public presentations are shown and compared between different globe browsing softwares.

\begin{center}
  \begin{table}
  \caption[]{Relevant features of different globe browsing softwares}
    \label{table:softwares}
  \resizebox{\textwidth}{!}{\begin{tabular}{| r | c | c | c | c | c | c |}
    \hline
                            & \textbf{Observable Universe scale} & \textbf{Focus on DOME configuration support} & \textbf{Ephemeris information integrated} & \textbf{Real world scientific data only} & \textbf{Free to use} & \textbf{Open source} \\ \hline

    Google Maps &  s  &  s  &  s  &  s  & s & s \\
    STK &  s  &  s  &  s  &  s  & s & s \\
    Uniview &  s  &  s  &  s  &  s  & s & s \\
    WWT &  s  &  s  &  s  &  s  & s & s \\
    Digistar &  s  &  s  &  s  &  s  & s & s \\
    DigitalSky &  s  &  s  &  s  &  s  & s & s \\
    Outerra &  s  &  s  &  s  &  s  & s & s \\
    Space Engine &  s  &  s  &  s  &  s  & s & s \\
    OpenSpace &  s  &  s  &  s  &  s  & s & s \\
    \hline
  \end{tabular}}
  \end{table}
\end{center}

\section{Tesselating the Ellipsoid}

Triangle models are still the most common way of modeling renderable objects in 3D computer graphics softwares, even though other rendering techniques such as volumetric ray casting also can be considered for terrain rendering \cite{cozzi11}.

A triangle mesh, or more generally a polygon mesh, is defined by a limited number of surface elements. This means that ellipsoids need to be approximated by some sort of tessellation or subdivision surface when modeled as a polygon mesh. There are several techniques for tessellating an ellipsoid. Some of them are covered in this section.

\subsection{Geographic Grid Tessellation}

Tessellating the ellipsoid using a geographic grid is a very straightforward approach. Ellipsoid vertex positions can be calculated using a transform from geographic (latitude and longitude) space to a cartesian coordinate space. An implementation of such a transform is implemented and described by Cozzi and Ring \cite[p. 25]{cozzi11}. Figure (fisk) shows three geographic grid tessellations of ellipsoids with constant number of latitudinal segments of 2, 4 and 8 respectively. A common issue with this approach is something referred to as polar pinching. At both of the poles, segments will be pinched to one point which leads to an increasing amount of segments per area. This in turn results in oversampling and visual artifacts in texture mapping due to the very thin quads.