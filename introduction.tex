\chapter{Introduction}

Scientific visualization of space research, also known as astro-visualization, works as an important tool for scientists to communicate their work in exploring the cosmos. 3D computer graphics has shown to be an efficient tool for bringing insights from geological and astronomical data, as spatial and temporal relations can intuitively be interpreted through 3D visualizations.

Researching and mapping celestial bodies other than the Earth is an important part of expanding the space frontier; virtually rendering these globes using real gathered map and terrain data is a natural part of any scientifically accurate space visualization software.

Important parts of a software for visualizing celestial bodies include the ability to render terrains together with color textures of various sources. The focus of this thesis is put on globe rendering using high fidelity geographical data such as texture maps, maps of scientific parameters, and digital terrain models. The globe rendering feature with the research involved was implemented for the software OpenSpace. The implementation was separated enough from the main program to avoid dependencies and make the thesis independent of specific implementation details. 

\section{Background}

\subsection{OpenSpace}

OpenSpace is an open-source, interactive data visualization software with the goal of bringing astro-visualization to the broad public and serve as a platform for scientists to talk about their research. The software supports rendering across multiple screens, allowing immerse visualizations on tiled displays as well as in dome displays using multiple projectors \cite{openspace}.

With a real time rendering software such as OpenSpace, the human curiosity involved in exploration easily becomes obvious when the user is given the ability to freely fly around in space and near the surface of other worlds and discover places they probably never can visit in real life. This is also true for public presentations where researchers such as geologists can go into details about their knowledge and showing it through scientific visualization.

An important part of the software is to avoid the use of procedurally generated data. This is to express where the frontier of science and exploration is currently at and how it progresses through space missions with the goal of mapping the Universe. A general globe browsing feature provides a means of communicating this progress through continuous mapping of planets, moons and dwarf planets within our solar system.

\subsection{Globe Browsing}

The term globe browsing can be described as exploration of geospatial data on a virtual representation of a globe. The word globe is a general term used to describe nearly elliptical celestial objects such as planets, moons, dwarf planets and asteroids.

Globe rendering with the purpose of multi-scale browsing has been used for quite some time in flight simulators, map services and astro-visualization. Prerendered flight paths were visualized as early as the late 1970s by NASA's Jet Propulsion Laboratory \cite{cozzi11}. 

Google Earth \cite{googlemaps} enables browsing of the Earth within a web browser using geometries for cities of high detail. The National Oceanic and Atmospheric Administration (NOAA) provides a sophisticated sphere rendering system, ''Science On a Sphere'', with the ability to visualize a vast amount of geospatial data on spheres with a temporal dimension \cite{sos}.

There are other commercial softwares that enables larger scale visualization of the Universe with real positional data gathered through research by the National Aeronautics and Space Administration (NASA), the European Space Agency (ESA) and others. Satellite Toolkit (STK) enables this by integrating ephemeris information through the SPICE interface \cite{spice} which allows accurate placing of celestial bodies and space crafts within our solar system using real measured data. Uniview from SCISS AB also enables SPICE integration with sophisticated rendering techniques and dome theatre support \cite{uniview}.

There are other significant globe browsing softwares used in dome theatres such as World Wide Telescope (WWT) \cite{wwt}, Evans \& Sutherland's Digistar \cite{digistar} and Sky-Skan's DigitalSky \cite{digitalsky}.

Other relevant softwares that currently do not support dome configuration rendering but none the less are very adequate in their techniques of integrating globe browsing and globe rendering include Outerra \cite{outerra} and Space Engine \cite{spaceengine}. Both focusing on merging real data with procedurally generated terrains where real data is not available.

Geographic information systems (GIS) are softwares with the purpose of gathering a wide range of geographic map data and visualizing it in various different ways. Even though most of these softwares use GIS features, many of them are not considered GIS. However, they all have the globe browsing feature in common. Technicalities in how it is implemented varies as their end target users are different.

In table \ref{table:softwares}, features relevant to globe browsing in public presentations are shown and compared between different visualization softwares that integrate globe browsing.

\begin{center}
  \begin{table}
  \caption[]{Relevant features of different globe browsing softwares}
    \label{table:softwares}
  \resizebox{\textwidth}{!}{\begin{tabular}{| r | c | c | c | c | c | c |}
    \hline
        & \textbf{Observable} & \textbf{Focus on Dome} & \textbf{Ephemeris} & \textbf{Scientific} & \textbf{Free} & \textbf{Open} \\ 
	& \textbf{Universe scale} & \textbf{configuration support} & \textbf{data integrated} & \textbf{data only} & \textbf{to use}  & \textbf{source} \\ \hline

    Google Maps &  &  &  & \ding{52} & \ding{52} & \\
    STK &  &  & \ding{52} & \ding{52} & & \\
    Uniview & \ding{52} & \ding{52} & \ding{52} & \ding{52} &  & \\
    WWT & \ding{52} & \ding{52} & \ding{52} & \ding{52} & \ding{52} & \ding{52} \\
    Digistar & \ding{52} & \ding{52} & \ding{52} & \ding{52} &  &  \\
    DigitalSky & \ding{52} & \ding{52} & \ding{52} & \ding{52} &  &  \\
    Outerra &  &  &  &  & \ding{52} & \\
    Space Engine & \ding{52}  &  & \ding{52} &  & \ding{52} & \\
    OpenSpace & Future Plan & \ding{52} & \ding{52} & \ding{52} & \ding{52} & \ding{52} \\
    \hline
  \end{tabular}}
  \end{table}
\end{center}

\section{Objectives}

The goal of this thesis is not only to provide OpenSpace with a globe browsing feature. There are some specific demands that play a significant role in the focus of this work. These are listed below:

\begin{enumerate}
    \item Ability to retrieve map data from the most common standardized web map data interfaces: WMS, TMS and WMTS.
    \item Ability to render height maps and color textures with up to 25 cm per pixel resolution
    \item Ability to layer multiple map datasets on top of each other. This makes it possible to visualize a range of scientific parameters on top of a textured globe
    \item Globe browsing should be done at interactive frame rates: at least 60 frames per second on modern consumer gaming graphics cards
    \item Correct positional mapping of objects rendered near the globe surface
    \item Support animation of map datasets with time resolution
    \item An intuitive interaction mode is also required to get the most out of globe browsing. It must support:
	\begin{enumerate}
    		\item Horizontal following of reference ellipsoid
		\item Decrease in sensitivity when closer to the surface of the globe
		\item Terrain following to avoid popping down under the surface
	\end{enumerate}
\end{enumerate}

\section{Delimitations}

To focus on the important aspects of globe browsing and its purposes for public presentations, some important delimitations had to be taken into consideration.

We do not consider rendering of globes with distances to the origin greater than the radius of our solar system. Direct imaging of exoplanets is far from usable for mapping and is mainly a method for locating planets \cite{exoplanets}. Since we don't yet have map data for exoplanets, and OpenSpace does not currently aim at producing procedurally generated content, visualization of exoplanets is not the focus for this thesis.

One important delimitation for the project is to limit the geometry of a globe to height mapped grids. This will make it possible to perform vertex blending on the GPU as well as simplify the implementation to a uniform method of rendering across the whole globe.

We will not focus on re-projecting maps between different georeferenced coordinate systems. Therefore the implementation must be limited to reading a specific map projection. Reprojecting maps can be considered future work to generalize the ability to read map data as well as optimizing rendering output and performance.

One goal of OpenSpace is to produce awe inspiring visual effects. The current state of the project requires the foundations to be in place and globe browsing is one of the main features that need to be implemented before real sophisticated rendering techniques can be developed. We will not consider rendering of atmospheres or water effects and we will not consider shading techniques that requires changing of the current rendering pipeline of OpenSpace such as deferred shading.

\section{Challenges}

There are multiple technical challenges to tackle when designing a virtual globe renderer. Cozzi and Ring \cite{cozzi11} define the main challenges as following:

\begin{itemize}

\item \textbf{Precision} - In order to render small scale details of a virtual globe and also dolly out to see multiple virtual globes within the solar system, the limited precision of computer arithmetics needs to be considered.
\item \textbf{Accuracy} - Modeling globes as spheres is usually not a very accurate approach, as many planets and moons that rotate have different polar and equatorial radii.
\item \textbf{Curvature} - The curved nature of globes implies some extra challenges as opposed to worlds modeled based on flat surfaces. The challenge includes finding a suitable 2D-parameterization for tessellation and mapping.
\item \textbf{Massive datasets} - It is usual for real world geographical datasets to be too large to fit in GPU memory, RAM and even local drives. Instead, data need to be fetched from remote servers on demand using a so called out-of-core approach to rendering.
\item \textbf{Multithreading} - The need for multithreading is necessary as the program needs to retrieve geographical data from multiple sources, while at the same time retain a steady frame rate for rendering.

\end{itemize}

Details in issues and proposed solutions to these challenges will be discussed throughout the thesis.

\section{Coordinate Systems}

When dealing in mapping and rendering of globes, conversion between different coordinate systems becomes important. The coordinate systems needed for considering in development are listed below:

\begin{itemize}
\item \textbf{Cartesian coordinates} - Normal right handed OpenGL coordinates. Can be either in model space, world space or camera space.

\begin{itemize}
\item \textbf{World space coordinates} - Coordinate space where the globe does not necessarily have to be in the origin. This, for example, can be the J2000 coordinate system which have the origin in the barycenter of the solar system.
\item \textbf{Model space coordinates} - Coordinate space where the center of mass of the globe is positioned in the origin. This, for example, can be the WGS84 reference frame for the Earth.
\item \textbf{Camera space coordinates} - Standard OpenGL camera space where $-z$ is in the view direction of the camera.
\end{itemize}

\item \textbf{Georeferenced coordinates (projection coordinates)} - Two or three dimensional coordinate space defining a position on the globe surface. This, for example, can be defined in geodetics ($latitude$ and $longitude$) with or without the inclusion of an $altitude$ coordinate.

\item \textbf{Raster coordinates} - Two dimensional coordinate space defined for a map or an image. Coordinates are \emph{line} and \emph{pixel}.

\item \textbf{Texture coordinates (uv coordinates)} - Two dimensional coordinate space defined for one sub section of a map known as a tile.

\end{itemize}
