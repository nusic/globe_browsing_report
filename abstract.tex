\vspace{2in}
\begin{abstract}

This thesis is the result of a project conducted at the American Museum of Natural History (AMNH), New York with the goal of implementing a planetary visualization software to aid in public presentations and bringing space science to the public.

The work is part of the development of the software OpenSpace, which is the result of a collaboration between Link�ping University, AMNH and the National Aeronatics and Space Administration (NASA) among others.

Based on objectives and deliminations, the software was developed to handle out-of-core rendering of multiple data mapped on virtual globes around our solar system. Challenges such as precision, accuracy, curvature and massive datasets were considered and the result is a globe renderer using a chunked level of detail approach to rendering. The software can render texture layers of various sort to aid in scientific visualization on top of height mapped grids which gives accurate results rendered at interactive frame rates.

\end{abstract} 
